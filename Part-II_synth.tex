\documentclass{article}

%% Packages 
\usepackage[english]{babel} % Language setting
\usepackage[a4paper,top=2cm,bottom=2cm,left=3cm,right=3cm,marginparwidth=1.75cm]{geometry} % Set page size and margins
\usepackage{amsmath}
\usepackage{graphicx}
\usepackage[colorlinks=true, allcolors=blue]{hyperref}
%%%%%%%%%%%%%%%%%%%%%%%%%%%%%%%%%%%%%%%%%%%%%%%%%%%%%%%%%%%%%%%%%%%%%%%%%%%%%%%%

%% Title
\title{Anonymization of data for open science in psychology: \\ 
       Part II — Synthetic data
}
%% Authors
\author{Jiří Novák \and 
        Matthias Templ \and 
        Carolin Strobl
        }
%% Start of article %%%%%%%%%%%%%%%%%%%%%%%%%%%%%%%%%%%%%%%%%%%%%%%%%%%%%%%%%%%%
%%%%%%%%%%%%%%%%%%%%%%%%%%%%%%%%%%%%%%%%%%%%%%%%%%%%%%%%%%%%%%%%%%%%%%%%%%%%%%%%
\begin{document}
\maketitle

%%%%%%%%%%%%%%%%%%%%%%%%%%%%%%%%%%%%%%%%%%%%%%%%%%%%%%%%%%%%%%%%%%%%%%%%%%%%%%%
%% Abstract %%%%%%%%%%%%%%%%%%%%%%%%%%%%%%%%%%%%%%%%%%%%%%%%%%%%%%%%%%%%%%%%%%%
\begin{abstract}
\textcolor{red}{(Problem)} Psychology as a field experienced a crisis caused by a lack of reproducibility. On the one hand, this caused distrust in the results, but on the other hand, it enabled the development of open science practices. One of the key parts of open science is open data, which must be well anonymized in order to be disseminated. 

Making more data openly available would enhance the transparency and accessibility of research. However, privacy concerns often prevent many datasets from being shared publicly. Despite this, there is a growing expectation for researchers to share their data with others for purposes such as review, reanalysis, and reuse.

\textcolor{red}{(Methodology)} In this paper, we present best practices of statistical disclosure control for psychologists. The paper is divided into two parts: the first part consists of traditional approaches, and the second part focuses on the modern approach of using synthetic data.
The modern approaches synthesise data so that it can be disseminated without revealing confidential information that may be associated with specific respondents. 

\textcolor{red}{(Main findings)} \\ 
\textcolor{red}{(Conclusion)} \\ 

\end{abstract}

\keywords{\textbf{Keywords:}
open science, confidentiality, reproducibility, anonymization, synthetic data
}

%%%%%%%%%%%%%%%%%%%%%%%%%%%%%%%%%%%%%%%%%%%%%%%%%%%%%%%%%%%%%%%%%%%%%%%%%%%%%%%
%% Introduction %%%%%%%%%%%%%%%%%%%%%%%%%%%%%%%%%%%%%%%%%%%%%%%%%%%%%%%%%%%%%%%
\section{Introduction}

%% Open science %%%%%%%%%%%%%%%%%%%%%%%%%%%%%%%%%%%%%%%%%%%%%%%%%%%%%%%%%%%%%%%
\textcolor{red}{Discuss the importance of open science and data sharing. (more lengthy on part I, part II paper citing part I paper)}

x

%% Replication crisis %%%%%%%%%%%%%%%%%%%%%%%%%%%%%%%%%%%%%%%%%%%%%%%%%%%%%%%%%
\textcolor{red}{Discuss the Replication crisis (more lengthy on part I, part II paper citing part I paper)}

x

%% Aim %%%%%%%%%%%%%%%%%%%%%%%%%%%%%%%%%%%%%%%%%%%%%%%%%%%%%%%%%%%%%%%%%%%%%%%%
\textcolor{red}{State the aim of the paper, which is to explain statistical disclosure methods and showcase an anonymization or synthetization approach on a psychological dataset}

The aim of this paper is to explore and elucidate the various statistical disclosure control
methods available for the anonymization of data in the context of open science within psychology.
This paper will specifically focus on presenting a detailed examination of data
synthetization techniques, demonstrating their application to a psychological dataset. Through this analysis, the paper seeks to provide practical insights into how these methods can be effectively utilized to balance the need for data transparency and privacy in psychological research.

%%%%%%%%%%%%%%%%%%%%%%%%%%%%%%%%%%%%%%%%%%%%%%%%%%%%%%%%%%%%%%%%%%%%%%%%%%%%%%%
%% Overview of SDC Methods %%%%%%%%%%%%%%%%%%%%%%%%%%%%%%%%%%%%%%%%%%%%%%%%%%%%
\section{Overview of SDC Methods}

\textcolor{red}{Overview of Anonymization or Synthetization Techniques, Description of SDC Methods}

x

\textcolor{red}{Review existing literature on statistical disclosure control methods (with focus on applications in psychology/psychometrics)}

x

\textcolor{red}{Discuss specific challenges and considerations or synthetization of data in psychology/psychometrics.}

x

\textcolor{red}{Discuss the possibilities on utility measurement}

x

\textcolor{red}{Discuss the disclosure scenarios (basically for anonymization: identity and attribute disclosure; for synthetic data: membership and inferential disclosure)}

x

%%%%%%%%%%%%%%%%%%%%%%%%%%%%%%%%%%%%%%%%%%%%%%%%%%%%%%%%%%%%%%%%%%%%%%%%%%%%%%%
%% Synthetization %%%%%%%%%%%%%%%%%%%%%%%%%%%%%%%%%%%%%%%%%%%%%%%%%%%%%%%%%%%%%
\section{Case Study: Synthetization a data set from psychology}

\textcolor{red}{Discuss criteria for selecting appropriate synthetization techniques for the case study.}

x

\subsection{Data}

The data for this example is from the Answers to the Machivallianism Test, a version of the MACH-IV from Christie and Geis~\cite{Data}, which comprises 73,489 records.
Includes variables about Likert-rated items and demographic/other items.

\subsection{sdcMicro}

Package sdcMicro~\cite{2024_Sdcmicro}

\subsection{synthpop}

Package synthpop~\cite{2022_Synthpop}

\subsection{simPop}

Package simPop~\cite{2022_Simpop}

\textcolor{red}{Showcase use of SDC methods}

x

\subsection{Case Study: Synthetization a data set from psychology}



%%%%%%%%%%%%%%%%%%%%%%%%%%%%%%%%%%%%%%%%%%%%%%%%%%%%%%%%%%%%%%%%%%%%%%%%%%%%%%%
%% Results %%%%%%%%%%%%%%%%%%%%%%%%%%%%%%%%%%%%%%%%%%%%%%%%%%%%%%%%%%%%%%%%%%%%
\section{Results}

\textcolor{red}{Utility Assessment}

x

\textcolor{red}{Disclosure Risk Assessment}

x

%% Discussion %%%%%%%%%%%%%%%%%%%%%%%%%%%%%%%%%%%%%%%%%%%%%%%%%%%%%%%%%%%%%%%%%
%%%%%%%%%%%%%%%%%%%%%%%%%%%%%%%%%%%%%%%%%%%%%%%%%%%%%%%%%%%%%%%%%%%%%%%%%%%%%%%
\section{Discussion}

\textcolor{red}{Recommendations and limits on the use of synthetization methods}

x

\textcolor{red}{Recommendations and limits on the use of utility assessment}

x

\textcolor{red}{Recommendations and limits on the use of disclosure risk measures}

x

\textcolor{red}{Challenges due to hierarchical data and cluster structures (e.g., all children in a class are surveyed).}

x

\textcolor{red}{Future research: longitudinal data}

x

%% Acknowledgment %%%%%%%%%%%%%%%%%%%%%%%%%%%%%%%%%%%%%%%%%%%%%%%%%%%%%%%%%%%%%
%%%%%%%%%%%%%%%%%%%%%%%%%%%%%%%%%%%%%%%%%%%%%%%%%%%%%%%%%%%%%%%%%%%%%%%%%%%%%%%
\section*{Acknowledgment \& Disclosure} 
\subsection*{Acknowledgment} 
This work was funded by the Swiss National Science Foundation with grant \textit{"Harnessing event and longitudinal data in industry and health sector through privacy preserving technologies"} (grant number 211751).

\subsection*{Disclosure of Interests} 
The authors have no competing interests to declare that are relevant to the content of this article. 

%% End of article %%%%%%%%%%%%%%%%%%%%%%%%%%%%%%%%%%%%%%%%%%%%%%%%%%%%%%%%%%%%%%
%%%%%%%%%%%%%%%%%%%%%%%%%%%%%%%%%%%%%%%%%%%%%%%%%%%%%%%%%%%%%%%%%%%%%%%%%%%%%%%%

%% References
\bibliographystyle{plain}
\bibliography{bib_part-II}
%%%%%%%%%%%%%%%%%%%%%%%%%%%%%%%%%%%%%%%%%%%%%%%%%%%%%%%%%%%%%%%%%%%%%%%%%%%%%%%%

\end{document}