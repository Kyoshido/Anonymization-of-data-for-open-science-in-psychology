\documentclass{article}

%% Packages 
\usepackage[english]{babel} % Language setting
\usepackage[a4paper,top=2cm,bottom=2cm,left=3cm,right=3cm,marginparwidth=1.75cm]{geometry} % Set page size and margins
\usepackage{amsmath}
\usepackage{graphicx}
\usepackage[colorlinks=true, allcolors=blue]{hyperref}
%%%%%%%%%%%%%%%%%%%%%%%%%%%%%%%%%%%%%%%%%%%%%%%%%%%%%%%%%%%%%%%%%%%%%%%%%%%%%%%%

%% Title
\title{Anonymization of data for open science in psychology: \\ 
       Part I — traditional anonymization
}
%% Authors
\author{Jiří Novák \and 
        Matthias Templ \and 
        Carolin Strobl
        }
%% Start of article %%%%%%%%%%%%%%%%%%%%%%%%%%%%%%%%%%%%%%%%%%%%%%%%%%%%%%%%%%%%
%%%%%%%%%%%%%%%%%%%%%%%%%%%%%%%%%%%%%%%%%%%%%%%%%%%%%%%%%%%%%%%%%%%%%%%%%%%%%%%%
\begin{document}
\maketitle

%%%%%%%%%%%%%%%%%%%%%%%%%%%%%%%%%%%%%%%%%%%%%%%%%%%%%%%%%%%%%%%%%%%%%%%%%%%%%%%
%% Abstract %%%%%%%%%%%%%%%%%%%%%%%%%%%%%%%%%%%%%%%%%%%%%%%%%%%%%%%%%%%%%%%%%%%
\begin{abstract}
\textcolor{red}{(Problem)} Psychology as a field experienced a crisis caused by a lack of reproducibility. On the one hand, this caused distrust in the results, but on the other hand, it enabled the development of open science practices. One of the key parts of open science is open data, which must be well anonymized in order to be disseminated. 

More openly available data would make research more transparent and accessible. Unfortunately, many datasets cannot be publicly available for privacy reasons. On the other hand, researchers are increasingly more expected to share data with others for review, reanalysis, and reuse.

\textcolor{red}{(Methodology)} In this paper, we present good practices of statistical disclosure control for psychologists. The paper is divided into two separated parts: the first part consists of traditional approaches, and the second part focuses on the modern approach of using synthetic data.
The traditional approaches modify data so that it can be disseminated without revealing confidential information that may be associated with specific respondents. 

\textcolor{red}{(Main findings)} \\ 

\textcolor{red}{(Conclusion)} \\ 

\end{abstract}

\keywords{
\textbf{Keywords:} 
open science, confidentiality, reproducibility, anonymization, sdc
}

%%%%%%%%%%%%%%%%%%%%%%%%%%%%%%%%%%%%%%%%%%%%%%%%%%%%%%%%%%%%%%%%%%%%%%%%%%%%%%%
%% Introduction %%%%%%%%%%%%%%%%%%%%%%%%%%%%%%%%%%%%%%%%%%%%%%%%%%%%%%%%%%%%%%%
\section{Introduction}

In contemporary research, Open science practices are becoming increasingly important. In their research, scientists process increasingly detailed data, which, however, are sensitive to the nature of the variables observed in the field of psychology.
For that reason, data cannot be shared well among researchers. This is one reason that caused the crisis of scientific study replicability. Data from publicly funded research should be more widely disseminated, at least among scientists. They should then be able to replicate studies, try new methods on the given data, or just verify the results presented in the given article or study.
\newline

%% Open science %%%%%%%%%%%%%%%%%%%%%%%%%%%%%%%%%%%%%%%%%%%%%%%%%%%%%%%%%%%%%%%
\textcolor{red}{Discuss the importance of open science and data sharing. (more lengthy on part I, part II paper citing part I paper)}

Open science is a movement that has been gaining strength and importance in recent years.
The movement's goal is to make available the results of scientific research, arising on the basis of public finances, so that they are reusable and replicable, traceable and transparent, trustworthy, and more financially effective, enabling a better connection of science across the world.
All started by Budapest Open Access Initiative~\cite{2012_OSI} in 2002, which was then supplemented with a set of rules in 2012~\cite{2012_OSI} and 2022~\cite{2022_OSI}.  This was followed by the Bethesda Statement on Open-Access Publishing~\cite{2003_Bethesda} in 2003 and Berlin Declaration on Open Access to Knowledge in the Sciences and Humanities~\cite{2003_Max_Planck}.

Budapest Open Access Initiative (BOAI)~\cite{2002_OSI} define \textit{Open Access} (OA) as \textit{"free availability on the public internet, permitting any users to read, download, copy, distribute, print, search, or link to the full texts of these articles, crawl them for indexing, pass them as data to software, or use them for any other lawful purpose, without financial, legal, or technical barriers other than those inseparable from gaining access to the internet itself"}.

In recommendations from 2012 BOAI~\cite{2012_OSI} further specified that \textit{"The worldwide campaign for OA to research articles should work more closely with the worldwide campaigns for OA to books, theses and dissertations, research data, government data, educational resources, and source code."}. In 2022, new recommendations~\cite{2022_OSI} for the next 10 years were released. Strong emphasis was put on Open infrastructure and its governance. It is recommended that OA texts, data, metadata, code, and other digital research outputs be hosted and published on open, community-controlled infrastructure. \textit{Open science} is made from open data, open metadata, open citations, open code, open protocols, open books, open theses and dissertations, open educational resources, open courseware, open digitization projects, open licenses, open standards, and open peer review.

From recent developments in recommendations to Open Science is necessary to mention Commission Recommendation (EU) 2018/790 of 25 April 2018 on access to and preservation of scientific information~\cite{2018_EU_2018/790}, the European University Association (EUA) Open Science Agenda 2025~\cite{2022_EUA} and UNESCO Recommendation on Open Science~\cite{2021_UNESCO}.

Recommendation 2018/790~\cite{2018_EU_2018/790} states that research data resulting from publicly funded research, including open access, should be findable, accessible, interoperable and re-usable, so-called \textit{FAIR principles}, unless this is unfeasible or conflicts with the future use of the research findings. There should be a strong emphasis on principle \textit{"As open as possible, as closed as necessary"}. 
EUA~\cite{2022_EUA} established its Open Science strategy for 2025 with three main priority areas: Open Access to scholarly outputs, FAIR research data, and institutional approaches to research assessment. In the vision of EUA 2025 are mentioned \textit{FAIR research data} as the norm in producing and sharing scientific knowledge and \textit{Open Science} as an integral part of research assessment practices.

UNESCO Recommendation~\cite{2021_UNESCO} defines \textit{Open Science} as \textit{"an
inclusive construct that combines various movements and practices aiming
to make multilingual scientific knowledge openly available, accessible and
reusable for everyone, to increase scientific collaborations and sharing of
information for the benefits of science and society, and to open the processes of scientific knowledge creation, evaluation and communication to societal actors beyond the traditional scientific community"}. In this recommendation, UNESCO promotes open access to scientific knowledge but equally emphasises the need for tools for pseudonymizing and anonymizing data so that as much data as possible can be shared as appropriate.
\newline

%% Anonymization %%%%%%%%%%%%%%%%%%%%%%%%%%%%%%%%%%%%%%%%%%%%%%%%%%%%%%%%%%%%%%
Anonymization of personal information must be approached from the point of view 
of the field of Statistical Disclosure Control (SDC). This research area is also known as Statistical disclosure limitation or Disclosure avoidance.
Hundepool~\cite{2012_Hundepool} describes SDC as a process that seeks to protect statistical data so that it can be released without divulging confidential information that can be linked to specific individuals or entities.

There are several major reasons for data anonymization, namely statistical principles, legal obligations, quality assurance, and ethical causes. 

United Nations~\cite{2015_UN} lists confidentiality of the data as a sixth fundamental principle of Official Statistics. This principle states that the statistical records of individual persons, businesses, or events used to produce Official Statistics are strictly confidential and to be used only for statistical purposes. It is evident that this principle applies not only to Official Statistics but also to every other field processing sensitive information, which should secure the confidentiality of its records. The European Union defined this approach in its Code of European Statistics~\cite{2018_Eurostat} as the fifth principle — Statistical Confidentiality and Data Protection, which states that the privacy of data providers, the confidentiality of the information they provide, its use only for statistical purposes and the security of the data are absolutely guaranteed.

Legislation imposes a legal obligation to protect individual business and personal data. Legal frameworks regulate what is allowed and what is not allowed regarding the publication of private information. In the member countries of the European Union, national statistical confidentiality is supported by EU legislation. The regulation of the European Parliament, better known by the abbreviation GDPR~\cite{2016_EU_2016/679}, is a pan-European legal framework for the protection of personal data, which protects the rights of citizens against unauthorized handling of their data and personal data.
In the context of the field of psychology, it is necessary to mention the legislation of the United States of America called HIPAA~\cite{1996_HIPAA} - Health Insurance Portability and Accountability Act.
The HIPAA Privacy Rule establishes standards for protection of individuals' medical records and other individually identifiable health information.

Quality assurance corresponds with confidence of respondents in the preservation of the confidentiality of individual information. If they do not trust in the confidentiality of the data, they may not provide accurate information. United Nations~\cite{2007_UN} emphasise that its necessary to maintain the trust of respondents if they are to continue to cooperate in their data collections. If respondents perceive that confidentiality of their data will not be protected, they are less likely to provide accurate data. 

Lastly disclosing information that can be linked to specific individuals or entities is unethical. Declaration on Professional Ethics~\cite{2010_ISI} set of Ethical Principles for statisticians and a wide array of creators and users of statistical data and tools. 
Disclose information that can be directly or indirectly linked to specific individuals or entities without their consent is considered unethical. Such actions may compromise privacy, lead to potential misuse of data, and violate principles of confidentiality. Ethical considerations require careful handling of sensitive information to prevent harm and uphold respect for personal and organizational boundaries. Necessary steps must be implemented to ensure that data are released in a way that protects the confidentiality of individuals, preventing their identities from being disclosed or inferred.
\newline

%% Replication crisis %%%%%%%%%%%%%%%%%%%%%%%%%%%%%%%%%%%%%%%%%%%%%%%%%%%%%%%%%
\textcolor{red}{Discuss the Replication crisis (more lengthy on part I, part II paper citing part I paper)}

The replication and reproducibility crisis has recently received a lot of attention. 
The replication crisis is a phenomenon in science, particularly in psychology and medicine, where many previously published scientific studies cannot be replicated or reproduced with similar results. This means that when other researchers attempt to repeat the experiments using the original methodology, they fail to achieve the same outcomes. This issue raises concerns about the reliability and validity of scientific findings, leading to criticism of certain research practices, such as statistical errors, flawed experimental design, or the publication of only positive results.

In 2015, Open Science Collaboration \cite{2015_OSC} of 271 authors examined the reproducibility of experiments in psychology. They selected about 100 studies from three psychology journals with the aim of achieving the same results as the original studies.  Only 36\% of original studies achieved significant results.


X
\newline

%% Aim %%%%%%%%%%%%%%%%%%%%%%%%%%%%%%%%%%%%%%%%%%%%%%%%%%%%%%%%%%%%%%%%%%%%%%%%
\textcolor{red}{State the aim of the paper, which is to explain statistical disclosure methods and showcase an anonymization or synthetization approach on a psychological dataset}

The aim of this paper is to explore and elucidate the various statistical disclosure control methods available for the anonymization of data in the context of open science within psychology. This paper will specifically focus on presenting a detailed examination of anonymization and data synthetization techniques, demonstrating their application to a psychological dataset. Through this analysis, the paper seeks to provide practical insights into how these methods can be effectively utilized to balance the need for data transparency and privacy in psychological research.

%%%%%%%%%%%%%%%%%%%%%%%%%%%%%%%%%%%%%%%%%%%%%%%%%%%%%%%%%%%%%%%%%%%%%%%%%%%%%%%
%% Overview of SDC Methods %%%%%%%%%%%%%%%%%%%%%%%%%%%%%%%%%%%%%%%%%%%%%%%%%%%%
\section{Overview of SDC Methods}

\textcolor{red}{Overview of Anonymization or Synthetization Techniques, Description of SDC Methods}

The goal of SDC methods is to find an optimal solution both in terms of the risk of disclosure and in terms of the utility of protected published data.

Methods intended to protect microdata are described in detail in the publications of Hundepool~\cite{2012_Hundepool}. In general, SDC methods can be divided according to when they are applied. The method can be applied directly to microdata, then we talk about pre-tabular methods, or to aggregated data in tables or hypercubes, and then we talk about post-tabular methods. The methods applied to microdata are naturally all pre-tabular methods.
We further distinguish the methods of modifying the values into three main groups: non-perturbative methods, perturbation methods, and methods of creating synthetic and hybrid data. Non-perturbative methods adjust the detail of the data display, perturbative methods add noise to the data, and synthetic and hybrid data generation methods generate new data based on the original data. 
\newline

\textcolor{red}{Review existing literature on statistical disclosure control methods (with focus on applications in psychology/psychometrics)}

X

\textcolor{red}{Discuss specific challenges and considerations or anonymizing of data in psychology/psychometrics.}

X

\textcolor{red}{Discuss the possibilities on utility measurement}

X

\textcolor{red}{Discuss the disclosure scenarios (basically for anonymization: identity and attribute disclosure; for synthetic data: membership and inferential disclosure)}

X

%%%%%%%%%%%%%%%%%%%%%%%%%%%%%%%%%%%%%%%%%%%%%%%%%%%%%%%%%%%%%%%%%%%%%%%%%%%%%%%
%% Synthetization %%%%%%%%%%%%%%%%%%%%%%%%%%%%%%%%%%%%%%%%%%%%%%%%%%%%%%%%%%%%%
\section{Case Study: Anonymizing a data set from psychology}

\textcolor{red}{Discuss criteria for selecting appropriate anonymization techniques for the case study.}

x

\textcolor{red}{Showcase use of SDC methods}

x

%%%%%%%%%%%%%%%%%%%%%%%%%%%%%%%%%%%%%%%%%%%%%%%%%%%%%%%%%%%%%%%%%%%%%%%%%%%%%%%
%% Results %%%%%%%%%%%%%%%%%%%%%%%%%%%%%%%%%%%%%%%%%%%%%%%%%%%%%%%%%%%%%%%%%%%%
\section{Results}

\textcolor{red}{Utility Assessment}

x

\textcolor{red}{Disclosure Risk Assessment}

To prove the success of data anonymisation, data utility is discussed as the main objective to be maximised while providing data with a disclosure risk below certain limits.

%% Discussion %%%%%%%%%%%%%%%%%%%%%%%%%%%%%%%%%%%%%%%%%%%%%%%%%%%%%%%%%%%%%%%%%
%%%%%%%%%%%%%%%%%%%%%%%%%%%%%%%%%%%%%%%%%%%%%%%%%%%%%%%%%%%%%%%%%%%%%%%%%%%%%%%
\section{Discussion}

\textcolor{red}{
\begin{itemize}
    \item introduction (review findings discuss outcomes; stake a claim)
    \item evaluation (analyze; offer explanations; reference the literature; state implications)    
    \item conclusion (limitations; recommendations)
\end{itemize}
} % end of red

%% Discussion %%%%%%%%%%%%%%%%%%%%%%%%%%%%%%%%%%%%%%%%%%%%%%%%%%%%%%%%%%%%%%%%%
%%%%%%%%%%%%%%%%%%%%%%%%%%%%%%%%%%%%%%%%%%%%%%%%%%%%%%%%%%%%%%%%%%%%%%%%%%%%%%%
\section{Discussion}

\textcolor{red}{Recommendations and limits on the use of anonymization methods}
   
x

\textcolor{red}{Recommendations and limits on the use of utility assessment}

x

\textcolor{red}{Recommendations and limits on the use of disclosure risk measures}

x

\textcolor{red}{Challenges due to hierarchical data and cluster structures (e.g., all children in a class are surveyed).}

x

\textcolor{red}{Future research: longitudinal data}

x

%% Acknowledgment %%%%%%%%%%%%%%%%%%%%%%%%%%%%%%%%%%%%%%%%%%%%%%%%%%%%%%%%%%%%%
%%%%%%%%%%%%%%%%%%%%%%%%%%%%%%%%%%%%%%%%%%%%%%%%%%%%%%%%%%%%%%%%%%%%%%%%%%%%%%%
\section*{Acknowledgment \& Disclosure} 
\subsection*{Acknowledgment} 
This work was funded by the Swiss National Science Foundation with grant \textit{"Harnessing event and longitudinal data in industry and health sector through privacy preserving technologies"} (grant number 211751).

\subsection*{Disclosure of Interests} 
The authors have no competing interests to declare that are relevant to the content of this article. 

%% End of article %%%%%%%%%%%%%%%%%%%%%%%%%%%%%%%%%%%%%%%%%%%%%%%%%%%%%%%%%%%%%%
%%%%%%%%%%%%%%%%%%%%%%%%%%%%%%%%%%%%%%%%%%%%%%%%%%%%%%%%%%%%%%%%%%%%%%%%%%%%%%%%

%% References
\bibliographystyle{plain}
\bibliography{bib_part-I}
%%%%%%%%%%%%%%%%%%%%%%%%%%%%%%%%%%%%%%%%%%%%%%%%%%%%%%%%%%%%%%%%%%%%%%%%%%%%%%%%

\end{document}